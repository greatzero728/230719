\begin{problem}{Хладнокровный дуб}{tree.in}{tree.out}{2 секунды}{}

%Автор задачи: Евгений Замятин
%Автор условия: Нияз Нигматуллин

Вампиры~--- странные существа. Они готовятся к проведению ежегодного праздника под названием <<Бессмертие>>. 

Для проведения праздника Эдварду было поручено подвесить за корень символ праздника~--- <<Хладнокровный дуб>>.

<<Хладнокровный дуб>> является обычным деревом: в нем $n$ вершин и $n - 1$ ребро, причем между любой парой 
вершин существует единственный простой путь, и корневая вершина имеет номер 1.

Из-за особенностей вампиров при нахождении под солнцем, главная часть праздника проводится ночью. Поэтому было 
решено использовать дуб в качестве освещения.
Для этого к каждой некорневой вершине дуба можно подвесить сколько угодно лампочек.
Для стабильности системы, к каждой лампочке проводят отдельную электрическую проводку из корневой вершины. 
Этот провод идет по пути дерева из корневой вершины
в вершину с соответствующей ей лампочкой, проходя через все промежуточные вершины и ребра на пути.

Эдвард никак не может решить, в какие вершины и сколько лампочек нужно повесить.

Вам задана структура <<Хладнокровного дуба>>, для каждой вершины $v > 1$ известна вершина $p_v$, за которую она подвешена. 
От вас требуется отвечать на странные вопросы Эдварда:
\begin{enumerate}
\item \texttt{add x y}~--- Эдвард вешает $y$ лампочек в вершину с номером $x$ ($2 \le x \le n$, $1 \le y \le 10^4$)
\item \texttt{del x y}~--- Эдвард убирает $y$ лампочек из вершины с номером $x$ ($2 \le x \le n$)
\item \texttt{ask x}~--- Эдвард просит посчитать $f(x)$ ($2 \le x \le n$)
\end{enumerate}

$f(x)$~--- это очень сложная функция, Эдварду не так просто ее описать, а о том, 
чтобы ее самому посчитать, он даже не задумывается. Вычисляется она так:

$$f(x) = \sum\limits_{v \in ST(x)}g(v)$$

$ST(x)$~--- это множество таких вершин, что, если в них повесить лампочку, то провод, который к ней придется провести,
будет проходить через ребро $(x,\,p(x))$. 

А $g(v)$~--- это число проводов, в данный момент проходящих через ребро $(v,\,p(v))$.

\InputFile
В первой строке задано натуральное число $n$ ($3 \le n \le 200\,000$)~--- число вершин в <<Хладнокровном дубе>>.

Во второй строке записано $n - 1$ чисел: $p_2, p_3 \ldots p_n$ ($1 \le p_i \le n$, $p_i \ne i$).

Гарантируется, что заданный <<Хладнокровный дуб>> является деревом.

В следующей строке задано натуральное число $m$ ($1 \le m \le 200\,000$)~--- число вопросов Эдварда.

В следующих $m$ строках заданы сами вопросы в формате, описанном в условии задачи.

Гарантируется, что все запросы корректны и Эдвард не будет убирать из вершины несуществующие лампочки.

\OutputFile

Для каждого вопроса Эдварда вида \texttt{ask x} выведите одно целое число~--- ответ на этот вопрос. 
Ответы требуется выводить в таком же порядке, в каком были заданы соответствующие им вопросы.

\Examples
\begin{example}%
\exmp{
2
1
4
add 2 1
ask 2
del 2 1
ask 2
}{
1
0
}%
\exmp{
5
1 2 2 1
12
add 3 4
ask 2
add 4 5
ask 2
del 3 1
ask 2
del 4 3
add 5 1
ask 5
del 5 1
ask 5
ask 2
}{
8
18
16
1
0
10
}%
\end{example}


\end{problem}
